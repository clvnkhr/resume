%! TEX root = ../cv-llt.tex

\begin{rubric}{Relevant Experience}
%\subrubric{Awards and Achievements}
  \entry*[2023] Experience with Python data science tools (\texttt{numpy}, \texttt{pandas}, \texttt{scikit-learn}, \texttt{matplotlib}, etc.)
  \entry*[2023] Contributor to the open source \textsc{Python} tool \href{https://github.com/bndr/pipreqs}{\texttt{pipreqs}}. 
  \entry*[2023] Developed open source neovim plugin \href{https://github.com/clvnkhr/macaltkey.nvim}{\texttt{macaltkey.nvim}} written in Lua.
  \entry*[2023] Contributed to the \href{https://oeis.org/A364354}{\textsc{OEIS}}.

\entry*[2023] \textbf{Parallel programming}, EPFL MOOC. In progress.
\entry*[2022]  \href{https://www.coursera.org/learn/scala-functional-programming/}{\textbf{Functional Programming Principles in Scala}},  \href{https://www.coursera.org/learn/scala-functional-program-design/}{\textbf{Functional Program Design in Scala}}, \href{https://www.coursera.org/learn/scala-akka-reactive/}{\textbf{Programming Reactive Systems}}, EPFL MOOC. Fully audited courses.
\entry*[2022] \href{https://certificates.cs50.io/61d7b5aa-582d-49e7-ada4-c7cd0b965c9b.pdf?size=letter}{\textbf{CS50x Introduction to Computer Science}}, Harvard University MOOC. \par Coding with \textsc{C, Python, Javascript, HTML, CSS, SQL} and \textsc{Flask}. Final project (\href{https://github.com/clvnkhr/tao2tex}{\texttt{tao2tex}}) on Github
\entry*[2015] Summer internship at PwC, Malaysia.
%

%\subrubric{Certification}
%\entry*[2014] \textbf{Certified XYZ Practioner}. Awarded by X Insitute.
%\entry*[2006] \textbf{Certified Level 3 in ABC}. Awarded by ABC.

\end{rubric}
